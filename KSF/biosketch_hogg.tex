\documentclass[12pt, fullpage, letterpaper]{article}
\usepackage{fancyheadings, graphicx}
\setlength{\headheight}{0ex}
\setlength{\headsep}{0ex}
%----- import some packages
\usepackage{graphicx, natbib, xcolor}

%----- set up some colors
\definecolor{darkgrey}{rgb}{0.3,0.3,0.3}
\definecolor{lightgrey}{rgb}{0.95,0.95,0.95}
\newcommand{\deemph}[1]{\textcolor{darkgrey}{\footnotesize{#1}}}

%----- exact 1-in margins
% NB: headheight and headsep MUST exist and be set
\setlength{\textwidth}{6.5in}
\setlength{\textheight}{9in}
\addtolength{\textheight}{-1.0\headheight}
\addtolength{\textheight}{-1.0\headsep}
\setlength{\topmargin}{0.0in}
\setlength{\oddsidemargin}{0.0in}
\setlength{\evensidemargin}{0.0in}

%----- typeset certain kinds of words
\newcommand{\acronym}[1]{{\small{#1}}}
\newcommand{\NYU}{\acronym{NYU}}
\newcommand{\AAS}{\acronym{AAS}}
\newcommand{\NASA}{\acronym{NASA}}
\newcommand{\NSF}{\acronym{NSF}}
\newcommand{\DOE}{\acronym{DOE}}
\newcommand{\EPRV}{\acronym{EPRV}}
\newcommand{\GP}{\acronym{GP}}
\newcommand{\PI}{\acronym{PI}}
\newcommand{\GRA}{\acronym{GRA}}
\newcommand{\LCDM}{\acronym{LCDM}}
\newcommand{\LSS}{\acronym{LSS}}
\newcommand{\CMB}{\acronym{CMB}}
\newcommand{\ESA}{\acronym{ESA}}
\newcommand{\ESO}{\acronym{ESO}}
\newcommand{\package}[1]{\textsf{#1}}
\newcommand{\project}[1]{\textsl{#1}}
\newcommand{\GALEX}{\project{\acronym{GALEX}}}
\newcommand{\LSST}{\project{\acronym{LSST}}}
\newcommand{\DESI}{\project{\acronym{DESI}}}
\newcommand{\SDSSIV}{\project{\acronym{SDSS-IV}}}
\newcommand{\eBOSS}{\project{e\acronym{BOSS}}}
\newcommand{\Kepler}{\project{Kepler}}
\newcommand{\Ktwo}{\project{\acronym{K2}}}
\newcommand{\HARPS}{\project{\acronym{HARPS}}}
\newcommand{\NEID}{\project{\acronym{NEID}}}
\newcommand{\NNEXPLORE}{\project{\acronym{NN-EXPLORE}}}
\newcommand{\foreign}[1]{\textsl{#1}}
\newcommand{\eg}{\foreign{e.g.}}
\newcommand{\etal}{\foreign{et~al.}}
\newcommand{\etc}{\foreign{etc.}}
\newcommand{\ie}{\foreign{i.e.}}
\newcommand{\vs}{\foreign{vs.}}

%----- math shih
\newcommand{\given}{\,|\,}

%----- units
\newcommand{\mps}{{\mathrm{m\,s^{-1}}}}
\newcommand{\kmps}{{\mathrm{km\,s^{-1}}}}
\newcommand{\Mpc}{{\mathrm{Mpc}}}

%----- typeset journals
\newcommand{\aj}{Astron.\,J.}
\newcommand{\apj}{Astrophys.\,J.}
\newcommand{\apjl}{Astrophys.\,J.\,Lett.}
\newcommand{\apjs}{Astrophys.\,J.\,Supp.\,Ser.}
\newcommand{\mnras}{Mon.\,Not.\,Roy.\,Ast.\,Soc.}
\newcommand{\pasp}{Pubs.\,Astron.\,Soc.\,Pac.}
\newcommand{\aap}{Astron.\,\&~Astrophys.}

%----- Tighten up section headings
\makeatletter
\renewcommand\section{\@startsection {section}{1}{\z@}%
                                   {-2.00ex \@plus -1ex \@minus -.2ex}%
                                   {1.0ex \@plus.2ex}%
                                   {\normalfont\large\bfseries}}
\renewcommand\subsection{\@startsection{subsection}{2}{\z@}%
                                     {-1.75ex\@plus -1ex \@minus -.2ex}%
                                     {1.0ex \@plus .2ex}%
                                     {\normalfont\normalsize\bfseries}}
\renewcommand\paragraph{\@startsection{paragraph}{4}{\z@}%
                                    {2.75ex \@plus1ex \@minus.2ex}%
                                    {-1em}%
                                    {\normalfont\normalsize\bfseries}}
\makeatother

%----- Tighten up paragraphs and lists
\setlength{\parskip}{0.0ex}
\setlength{\parindent}{0.2in}
\setlength{\bibsep}{0pt}
\renewenvironment{itemize}{\begin{list}{$\bullet$}{%
  \setlength{\topsep}{0.0ex}%
  \setlength{\parsep}{0.0ex}%
  \setlength{\partopsep}{0.0ex}%
  \setlength{\itemsep}{0.0ex}%
  \setlength{\leftmargin}{2.0\parindent}}}{\end{list}}
\newcounter{actr}
\renewenvironment{enumerate}{\begin{list}{{\textbf{\textsf{\arabic{actr}.}}}}{%
  \usecounter{actr}%
  \setlength{\topsep}{0.0ex}%
  \setlength{\parsep}{0.0ex}%
  \setlength{\partopsep}{0.0ex}%
  \setlength{\itemsep}{0.0ex}%
  \setlength{\leftmargin}{2.0\parindent}}}{\end{list}}

%----- Special Hogg list for references
\newcommand{\hogglist}{%
  \rightmargin=0in
  \leftmargin=1em
  \topsep=0ex
  \partopsep=0pt
  \itemsep=0ex
  \parsep=0pt
  \itemindent=-1.0\leftmargin
  \listparindent=\leftmargin
  \settowidth{\labelsep}{~}
  \usecounter{enumi}
}

%----- mess with paragraph spacing!
\makeatletter
\renewcommand\paragraph{\@startsection{paragraph}{4}{\z@}%
                                    {1ex}%
                                    {-1em}%
                                    {\normalfont\normalsize\bfseries}}
\makeatother

%----- side-to-side figure macro
%------- make numbers add up to 94%
 \newlength{\figurewidth}
 \newlength{\captionwidth}
 \newcommand{\ssfigure}[3]{%
   \setlength{\figurewidth}{#2\textwidth}
   \setlength{\captionwidth}{\textwidth}
   \addtolength{\captionwidth}{-\figurewidth}
   \addtolength{\captionwidth}{-0.02\figurewidth}
   \begin{figure}[htb]%
   \begin{tabular}{cc}%
     \begin{minipage}[c]{\figurewidth}%
       \resizebox{\figurewidth}{!}{\includegraphics{#1}}%
     \end{minipage} &%
     \begin{minipage}[c]{\captionwidth}%
       \textsf{\caption[]{\footnotesize {#3}}}%
     \end{minipage}%
   \end{tabular}%
   \end{figure}}

%----- top-bottom figure macro
 \newlength{\figureheight}
 \setlength{\figureheight}{0.75\textheight}
 \newcommand{\tbfigure}[2]{%
   \begin{figure}[htp]%
   \resizebox{\textwidth}{!}{\includegraphics{#1}}%
   \textsf{\caption[]{\footnotesize {#2}}}%
   \end{figure}}

%----- deal with pdf page-size stupidity
\special{papersize=8.5in,11in}
\setlength{\pdfpageheight}{\paperheight}
\setlength{\pdfpagewidth}{\paperwidth}

% no more bad lines, etc
\sloppy\sloppypar\raggedbottom\frenchspacing

\renewcommand{\headrulewidth}{0pt}
\pagestyle{empty}

\begin{document}

\subsection*{David W. Hogg}
\noindent http://cosmo.nyu.edu/hogg/

\subsubsection*{Professional preparation}
\begin{list}{}{\hogglist}
\item
SB (Physics) 1992, Massachusetts Institute of Technology
\item
PhD, Physics, 1998, California Institute of Technology
\item
Long-term member, 1997--2001, Institute for Advanced Study
\end{list}

\subsubsection*{Appointments}
\begin{list}{}{\hogglist}
\item
\deemph{current:} Professor of Physics and Data Science, New York University, 2014--.
\item
\deemph{current:} Group Leader, Astronomical Data Group, Center for Computational Astrophysics,
Flatiron Institute, New York, 2017--.
\item
\deemph{current:} Adjunct Senior Staff Scientist, Max-Planck-Institut f\"ur Astronomie,
Heidelberg, Germany, 2012--.
\item
Consultant, Flatiron Institute, New York, 2016--2017.
\item
Consultant, Simons Center for Data Analysis,
Simons Foundation, New York, 2015--2016.
\item
Associate Professor of Physics with tenure, New York University, 2007--2014.
\item
Visiting Scientist, Max-Planck-Institut f\"ur Astronomie,
Heidelberg, Germany, 2006--2011.
\item
Visiting Professor, Department of Astronomy and Astrophysics, Columbia
University, 2008
\item
Assistant Professor of Physics, New York University, 2001--2007.
\end{list}

\subsubsection*{Products---relevant} % 5

\subsubsection*{Products---other significant} % 5

\subsubsection*{Synergistic activities} % up to 5
\begin{list}{}{\hogglist}
\item
Helped the astronomical community through advisory committees:
\project{Spitzer} Science Center Oversight Committee (to advise the \project{Spitzer Space Telescope} project) 2008--present.
\project{\NASA\ Extragalactic Database} Users Committee 2006--2013.
Astronomy and Astrophysics Advisory Committee (established by the US
Congress to advise \NSF, \NASA, and \DOE\ on areas of overlap) 2014--2017.
\item
Photometrically calibrated the \project{Sloan Digital Sky Survey} imaging data; this
calibration has been used in all data releases since DR8.
Involved in the every phase of the
\project{SDSS-IV} project. Member of the
\project{SDSS-IV} Collaboration Council,
2013--present. Currently helping with the design of \project{SDSS-V}.
%% \item
%% Served in administrative roles at NYU, including
%% Director of Undergraduate Studies, Department of Physics, New York University, 2008--2015; and
%% Executive Director of the \textsc{usd}12.6M \project{Moore--Sloan Data Science Environment at \textsc{nyu}}, 2013--2015.
\item
Supervised the production of multiple multi-user open-source software projects,
including \project{Astrometry.net} for blind image calibration (thousands of users),
\project{Daft} for drawing graphical models, and the \project{emcee}
MCMC sampler (more than 3500 citations in 6 years).
\item
Wrote and maintain pedagogical documents about a variety of topics including
cosmological distance measures (arXiv:astro-ph/9905116),
probability (arXiv:1205.4446),
fitting models (arXiv:1008.4686), and
MCMC (arxiv:1710.06068).
\item
Designed, organized, and promoted hack events for education and scientific research:
Co-founded and co-organized the \project{AstroHackWeek} series.
Founded and co-organized the \project{Hack Together Day} at the Winter AAS meetings.
Founded, designed, and co-organized the \project{Gaia Sprints}.
Co-authored a paper on some of these activities (arXiv:1711.00028).
\end{list}

\end{document}
